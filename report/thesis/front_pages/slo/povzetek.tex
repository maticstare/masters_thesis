%---------------------------------------------------------------
% SLO: slovenski povzetek
% ENG: slovenian abstract
%---------------------------------------------------------------
\selectlanguage{slovene} % Preklopi na slovenski jezik
\addcontentsline{toc}{chapter}{Povzetek}
\chapter*{Povzetek}

\noindent\textbf{Naslov:} \ttitle
\bigskip

V magistrskem delu predstavljam metodo za zaznavanje in preprečevanje trkov med vlakom in predorom, ki združuje obdelavo oblaka točk s parametričnim modeliranjem gibanja vagona. Predor pretvorimo v horizontalne plasti, stene aproksimiramo z B-zlepki, gibanje vagona pa simuliramo vzdolž kontrolnih točk z metodo pozicioniranja glede na osi. Za zaznavanje trkov na vsaki višini vzorčimo šest kritičnih robnih točk in izračunamo razdalje do geometrije predora. Optimizacijski postopek nato določi največji dopustni model vagona. Sistem podpira prileganje različnih oblik tovora in nudi interaktivno 3D vizualizacijo. Glavni prispevki so detektor trkov, ki upošteva geometrijo predora, postopek za izpeljavo največjega varnega modela vagona ter orodja za prileganje tovora in vizualizacijo.

\subsection*{Ključne besede}
\textit{\tkeywords}
\clearemptydoublepage